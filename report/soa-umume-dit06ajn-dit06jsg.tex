% -*- coding: utf-8 -*-
% \documentclass[journal]{IEEEtran}
\documentclass[titlepage, twocolumn, a4paper, 10pt]{article}
\usepackage{parskip}
\usepackage[english]{babel}
\usepackage[utf8]{inputenc}
\usepackage{verbatim}
\usepackage{fancyhdr}
\usepackage{graphicx}
\usepackage{url}
\usepackage{varioref}

%%%%%%%%%%%%%%%%
% Column spacing
% \setlength{\columnsep}{7mm}
\renewcommand{\sfdefault}{phv}
\renewcommand{\rmdefault}{ptm}
\renewcommand{\ttdefault}{pcr}

\hyphenpenalty=750
% If we didn't adjust the interword spacing, 2200 might be better.
% The TeX default is 1000
\hbadness=1350
% IEEE does not use extra spacing after punctuation
\frenchspacing

% V1.7 increase this a tad to discourage equation breaks
\binoppenalty=1000 % default 700
\relpenalty=800     % default 500


% margin note stuff
\marginparsep      10pt
\marginparwidth    20pt
\marginparpush     25pt


% if things get too close, go ahead and let them touch
\lineskip            0pt
\normallineskip      0pt
\lineskiplimit       0pt
\normallineskiplimit 0pt

\topmargin    -49.0pt
\headheight   12pt
\headsep      0.25in

\textheight       58pc  % 9.63in, 696pt
\columnsep         1pc
\textwidth        42pc   % 2 x 21pc + 1pc = 43pc

% the default side margins are equal
\oddsidemargin        0.680in
\evensidemargin       0.680in
% compensate for LaTeX's 1in offset
\addtolength{\oddsidemargin}{-1in}
\addtolength{\evensidemargin}{-1in}
\topmargin        -0.25in
% we retain the reserved, but unused space for headers
\addtolength{\topmargin}{-\headheight}
\addtolength{\topmargin}{-\headsep}

%%%%%%%%%%%%%%%%


\usepackage[pdfborder={0 0 0 0}]{hyperref}

% Include pdf with multiple pages ex \includepdf[pages=-, nup=2x2]{filename.pdf}
\usepackage[final]{pdfpages}

% Place figures where they should be use [H]
\usepackage{float}

% Float for text
\floatstyle{ruled}
\newfloat{code}{!htb}{lop}
\floatname{code}{CodeSnippet}

% vars
\def\title{Umume}
\def\preTitle{A RESTful service}
\def\kurs{Service-Oriented Architectures, HT-09}


\def\namn{Anton Johansson}
\def\mail{dit06ajn@cs.umu.se}

\def\namnTva{Jonny Strömberg}
\def\mailTva{dit06jsg@cs.umu.se}


\def\pathtocode{\url{dit06ajn~/edu/soa/umume}}
\def\handledareEtt{P-O Östberg, p-o+soa@cs.umu.se}

\def\inst{Computer Science}
\def\dokumentTyp{Report}

\begin{document}
\begin{titlepage}
  \thispagestyle{empty}
  \begin{small}
    \begin{tabular}{@{}p{\textwidth}@{}}
      UMEÅ UNIVERSITY \hfill \today \\
      Department of \inst \\
      \dokumentTyp \\
    \end{tabular}
  \end{small}
  \vspace{10mm}
  \begin{center}
    \LARGE{\preTitle} \\
    \huge{\textbf{\kurs}} \\
    \vspace{10mm}
    \LARGE{\title} \\
    \vspace{15mm}
    \begin{large}
      \namn, \mail \\
      \namnTva, \mailTva\\
      % Code-path: \texttt{\pathtocode}\\
      \vspace{10mm}
      Start-server: \texttt{\pathtocode > ./start-server.sh}\\
      Start-gui: \texttt{\pathtocode > ./start-gui.sh}
    \end{large}
    \vfill
    \large{\textbf{Supervisors}}\\
    \mbox{\large{\handledareEtt}}\\
  \end{center}
\end{titlepage}

\newpage
\mbox{}
\vspace{70mm}
\begin{center}
  % Dedication goes here
\end{center}
\thispagestyle{empty}
\newpage

\pagestyle{fancy}
\rhead{\today}
\lhead{\footnotesize{\namn, \mail\\\namnTva, \mailTva}}
\chead{}
\lfoot{}
\cfoot{}
\rfoot{}

\cleardoublepage
\newpage
\onecolumn
\tableofcontents
\twocolumn
\cleardoublepage

\fancyfoot[LE,RO]{\thepage}
\pagenumbering{arabic}

\section{Introduction}\label{sec:intro}
% Beskriv med egna ord vad uppgiften gick ut på. Är det någonting som
% varit oklart och ni gjort egna tolkningar så beskriv dessa.
This report describes the work done designing and implementing a RESTful web service providing information about employees and students at Umeå university. For this web service a client web site is done to highlight possible uses of the web service.

The main idea with the web service is to combine information from the existing information about every employee/student with a possibility for persons to add information about themselves. This information can for example be a brief description, visiting location and usernames for other web services such as Twitter\footnote{\url{http://twitter.com/}} to enable mashup implementations.

The web service will hereby be called \textit{Umume-rest}, a combination of \textit{Umu} as in Umeå university and {me} because it provides a service for students and employees. The client web site is called \textit{Umume-website}.

These implementations are done with the programming language
\textit{Java}\footnote{\url{http://java.sun.com/}} and uses the framework \textit{Spring Framework}\footnote{\url{http://www.springframework.org/}} for the web site, and \textit{Jersey}\footnote{\url{https://jersey.dev.java.net/}} for the RESTful Web service.

The original project specification for this work is shown in Appendix \ref{app:ps}.

\section{Problem analysis}\label{sec:problem-analysis}
% As this project emphasizes analysis and investigation of a loosely
% specified problem, include any assumptions you made during the
% analysis phase in your report. Also discuss problems encountered and
% alternative solutions considered in the analysis. The report should
% also discuss to what extent the requirement list is fulfilled, as
% well as to which extent you could adhere to the the project plan.
Since the web service 

\subsection{Member failures}\label{sec:member-failures}
A member of a group is considered to have failed only when it throws a
\textit{RemoteExceptioin}\footnote{\url{http://java.sun.com/javase/6/docs/api/java/rmi/RemoteException.html}}
as defined by \textit{Java RMI}. This means that \textit{GCom} makes
no guarantee about the time it takes to send a message to a group.
This guarantee could be achieved simply by changing the definition of
a member failure to include a time-limit for message delivery.

\subsection{Group discovery}\label{sec:group-discovery}
When a process wants to communicate with other processes using
\textit{GCom} there must be a way to find groups and group members
already existing. That starting point is defined by a global address
known by all \textit{GCom} members. This starting point will contain a
service for group discovery, described in more detail in section
\ref{sec:group-name-system}.

% TODO: Discuss single point of failure?

\section{Usage}\label{sec:usage}
% Förklara var programmet och källkoden ligger samt hur man kompilerar,
% startar och använder det. Förklara även översiktligt vad som händer
% när man använder de olika kommandona. Det räcker alltså inte att
% skriva "man skriver 'ant' för att kompilera", utan det måste även ingå
% en liten förklaring om vad som egentligen händer när man kör ant och
% varför det fungerar. Använd Internet eller litteratur för att själva
% ta reda på den information ni tycker känns relevant, dels för
% rapportens skull och dels för er egen. Kom ihåg att skriva tydliga
% (vetenskapliga) referenser!
All files needed to use set up the web service are located at:\\
\texttt{\pathtocode}/umume-rest and the files for the web site are at:\\
\texttt{\pathtocode}/umume-website

\subsercion{Servlet container}\label{sec:usage-servlet-container}
This project has been developed with \textit{Apache Tomcat 6.0.20}\footnote{http://tomcat.apache.org/} as servlet container but any Java servlet 
container would do.

\subsection{Web service - /umume-rest}
This catalog contains the following sub directories:
\begin{itemize}
\item \verb!src! contains the source code.
\item \verb!src/main/resources/! contains the SQLite database and 
the configuration
  files for standard behaviour of the compiled system, see section
  \ref{sec:configuration}. 
\item \verb!target! will, after a successful compilation,
  contain all the compiled sources as well as configuration files used
  by this web service.
\item \verb!lib! contains all requires third-party libraries
  needed by the \textit{Umume web service}, se section \ref{sec:required-libraries}.
\item \verb!conf! contains the build-properties file and the jar-file for Ivy.
\end{itemize}

And these are the files in the catalog:
\item \verb!build.xml! contains settings for building this web service.
\item \verb!ivy.xml! contains all Ivy dependencies. 
\item \verb!ivysettings.xml! contains the Ivy settings. 
\end{itemize}

\subsubsection{Compilation}\label{sec:compilation-rest}
The following commands will require the software tool \textit{Apache
  Ant}\footnote{http://ant.apache.org/}. More details about what
happens using \textit{ant} in this project is found in the file
\textit{build.xml}\footnote{http://ant.apache.org/manual/using.html}.

To compile \textit{umume-rest} issue the following command:\\
\begin{footnotesize}
  \verb!salt:./umume-rest> ant deploy!
\end{footnotesize}\\
This will create a directory \verb!target! if it does not already exists
and compile/move source-code and configuration files to that
directory. Then a war file is generated and moved to the servlet 
container (remember to configure the build properties in 
\verb!conf/build.properties!)

The root-directory for class-files when using \textit{GCom} is
compiled to \textit{target/classes}, while the root-directory for
test-code is compile to \textit{bin/test/java}.

To create \textit{jar}-file of the compiled sources issue the
following command:\\
\begin{footnotesize}
  \verb!salt:./GCom> ant jar!
\end{footnotesize}\\
This will create \textit{GCom.jar} which can be used when developing
in third party software or directly as a \textit{GNS}-server (see
section \ref{sec:group-name-system}) by
running:\\
\begin{footnotesize}
  \verb!salt:./GCom> java -jar GCom.jar!
\end{footnotesize}

\begin{code}
  \begin{footnotesize}
\begin{verbatim}
# Used by GNS
gcom.gns.port=1078

# FIFO, TOTAL_ORDER, NO_ORDERING,
# CASUAL_ORDERING, CASUALTOTAL_ORDERING
gcom.ordering=FIFO

# BASIC_MULTICAST, RELIABLE_MULTICAST
gcom.multicast=RELIABLE_MULTICAST
\end{verbatim}
  \end{footnotesize}
  \caption{applications.properties}\label{code:app-prop}
\end{code}


\section{System description}\label{sec:system}
% Beskriv översiktligt hur programmet är uppbyggt och hur det löser
% problemet.

% The GCom middleware consists of three (logical) modules, the group
% management module, the communication module and the message ordering
% module. These are, respectively, responsible for handling group
% membership issues, communication message exchange semantics and
% message (re)ordering issues. All of these modules need to function
% properly in order for your system to be able to ensure correct
% message delivery semantics.

%\begin{figure*}[!thb]
%  \centerline{\includegraphics[width=110mm]{images/Stack.pdf}}
%  \caption{GCom stack}
%  \label{fig:images/Stack}
%\end{figure*}


\subsubsection{Error handling}\label{sec:error-handling}
% To detect errors: The module monitors a group and indicates when a
% member of the group crashes (or for some other reason become
% unreachable).
When sending messages a group member may detect that one of the
receiving members has crashed. If the detecting member is a group
leader it will directly send a groupchange message, otherwise it will
send a membercrash message which when received by the group leader
will result in a groupchange message multicasted to the group. A
groupchange message contains information about the complete new group
composition whereas a membercrash message only contains information
about the members that have crashed.

When the member that has crashed is the group leader the exact same procedure is used except that when processes receive membercrash messages they will check if they are the new group leader, and if they are they will send a groupchange message to the group. Group members test if they are the new group leader by checking if they have the maximum value of \textit{UUID}s in the current group. This is a variant of the \textit{Bully alghoritm}, see page 482
% TODO: this section maybe should have contained info about normal leave
% messages, which we have not implemented.
% \subsubsection{Group changes}\label{sec:group-changes}
% To notify changes in group membership: The module notifies all group
% members about changes in group composition.


\begin{table}[H]
  \centering
  \begin{footnotesize}
    \begin{tabular} {c | c | c | c}
      % BEGIN RECEIVE ORGTBL casualtotal
      & p1 & p2 & p3 - Sequencer (hold all messages) \\
      \hline
      Send order & one &  &  \\
      & two &  &  \\
      \hline
      &  &  & (Release messages in reverse order) \\
      \hline
      Receive order & one & one & one \\
      & two & two & two \\
      % END RECEIVE ORGTBL casualtotal
    \end{tabular}
  \end{footnotesize}
  \caption{Casual-Total ordering tests}
  \label{tbl:castot}
\end{table}
\begin{comment}
  #+ORGTBL: SEND casualtotal orgtbl-to-latex :splice t
  |               | p1  | p2  | p3 - Sequencer (hold all messages)  |
  |---------------+-----+-----+-------------------------------------|
  | Send order    | one |     |                                     |
  |               | two |     |                                     |
  |---------------+-----+-----+-------------------------------------|
  |               |     |     | (Release messages in reverse order) |
  |---------------+-----+-----+-------------------------------------|
  | Receive order | one | one | one                                 |
  |               | two | two | two                                 |
\end{comment}


\section{Discussion}\label{sec:discussion}
% Vilka problem och begränsningar har din lösning av uppgiften? Hur
% skulle de kunna rättas till?
The following sections will discuss some of the problems and solutions
encountered while implementing \textit{GCom}.

% \section{Reflektioner}\label{Reflektioner}
% % Reflektioner - Var det något som var speciellt krångligt? Vilka
% % problem uppstod och hur löste ni dem? Vilka verktyg använde ni? Hur
% % upplevde ni de verktygen? + Allmänna synpunkter. Om ni har upplevt
% % problem på grund av olika miljöer (i termer av operativsystem och
% % liknande) så kan det även vara intressant att nämna det, samt motivera
% % ert val av miljö.

\subsection{No automatic failure detection}\label{sec:no-automatic-failure-detection}
We have no automatic "ping-function" in our system. That means that if for example
a group member crashed, no one will realize this before they tries to send a message. This also
implies that if the leader crashes and no messages is sent, there will be no way to connect to the group.

\subsection{Security}\label{sec:security}
Here there are great opportunities for development. For instance, theoreticly anyone could send I 
GroupChange-message and noone will know it the sender is the leader or not. It is the same way 
with all other messages types.

\section{Scenarios}\label{sec:scenarios}
This section shows different scenarios that explains how the system
works.


\section{Tests}\label{sec:tests}
% Noggranna testkörningar där man ser att programmet fungerar som det
% ska.

% During your demo, you will need to convince the teachers that your
% implementation works. Bring a test protocol, i.e., a series of tests
% that clearly demonstrates that your GCom fulfills the requirements
% and a test tool which can be used to apply it. The test protocol
% should include, e.g., tests of all message orders and multicast
% types. Bring a copy of the test protocol on paper, see page 491 in
% [DS] for suggested notation. Your test protocol must clearly state
% your names, user names, and which level you intend to demonstrate.

% The fact that a system cannot be formally proven to work does not
% make it impossible to implement - consider for example the Internet.
% Read pages 498 and 508 in [DS].



%%%%%%%%%%%%%%%% END APPENDIX AND STUFF %%%%%%%%%%%%%%%%
%\bibliographystyle{alpha}
%\bibliography{books.bib}

\newpage
\appendix
\pagenumbering{roman}
\section{Appendix}\label{sec:app}
% % Källkoden ska finnas tillgänglig i er hemkatalog
% % ~/edu/apjava/lab1/. Bifoga även utskriven källkod.
\subsection{Project Specification}\label{app:ps}
\end{document}
